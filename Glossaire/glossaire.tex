% !TEX root = ../flore.tex

\chapter{Glossaire illustré}

Les mots définis dans le glossaire sont marqués en \textcolor{teal}{vert sombre} dans le texte. Il suffit de cliquer dessus pour être amené à leur définition. De même, au sein du glossaire, les exemple sont en \textcolor{violet}{violet} et amène à l'espèce ou au genre concerné.

\ifpapieraquatre
\begin{multicols}{2}
\fi
\ifpapieracinq
\begin{multicols}{2}
\fi

\section*{A}

\entreeglossaire{Acaule}{plante sans tige aérienne apparente ou à tige si courte que les feuilles semblent naître de la racine : violette odorante, gentiane acaule.}

\entreeglossaire{Accrescent}{organe continuant à végéter et à s'accroître après la floraison :

\noindent
\begin{minipage}{\columnwidth}
\centering
\framebox[150pt]{\rule{0pt}{100pt}}
\end{minipage}
calice du Physalis, de l'Androsacace maxima, styles de la Benoîte.}

\entreeglossaire{Arbuste}{petit arbre de \SIrange{1}{5}{\meter}, à tige se ramifiant ordinairement dès la base : fusain, nerprun, troëne, genevrier.}

\section*{G}

\entreeglossaire{Glabre}{dépourvu de poils : buis, chou, persil.}

\section*{H}

\entreeglossaire{Herbacé}{vert ou ayant la consistance molle de l'herbe, par opposition à coloré ou à ligneux.}

\section*{P}

\entreeglossaire{Pubescent}{garni de poils fins, mous, courts et peu serrés : Delphinium pubescens.}

\section*{S}

\entreeglossaire{Sarmenteux}{tige ou rameaux ligneux, flexibles, faibles, ayant besoin d'un appui : vigne, clématite.}

\section*{V}

\entreeglossaire{Volubile}{tige qui s'enroule autour des corps voisins : houblon, cuscute, liseron, haricot.}

\ifpapieraquatre
\end{multicols}
\fi
\ifpapieracinq
\end{multicols}
\fi