% !TEX root = ../flore.tex

% a4
\ifpapieraquatre
\usepackage[
a4paper,
margin=15mm,
includefoot,
headheight=18pt
]{geometry}
\fi

% a5
\ifpapieracinq
\usepackage[
a5paper,
margin=10mm,
includefoot,
headheight=18pt
]{geometry}
\fi

% a6
\ifpapierasix
\usepackage[
a6paper,
margin=5mm,
includefoot,
headheight=18pt,
footskip=8mm
]{geometry}
\fi

% a7
\ifpapierasept
\usepackage[
paperwidth=74mm,
paperheight=105mm,
margin=2mm,
includefoot,
headheight=18pt,
footskip=8mm
]{geometry}
\fi

% Style xindy
\usepackage{filecontents}
\begin{filecontents*}{style.xdy}
;;; xindy style file
(markup-keyword-list :close "\dotfill")
(markup-locclass-list)
\end{filecontents*}

% français
\usepackage[frenchb]{babel}
\usepackage[french]{translator}

% fonte
\usepackage{libertine}
\usepackage[babel]{microtype}

% code lua
\usepackage{luacode}

% graphiques
\usepackage{graphicx}
\graphicspath{{Images/}}
\usepackage{subfig}
\usepackage{calc}

% couleurs
\usepackage[svgnames]{xcolor}

% index
\usepackage[xindy]{imakeidx}

% if then else
\usepackage{xifthen}

% compteurs
\usepackage{fmtcount}

% unités
\usepackage{siunitx}
\sisetup{detect-all,output-decimal-marker={,},list-final-separator={ et },range-phrase={ à },allow-number-unit-breaks,text-celsius = $^\circ\mkern-1mu$C,inter-unit-product={}\cdot{},per-mode=symbol,range-units=single}

% pdftooltip
\usepackage{pdfcomment}

% légendes figures
\usepackage{caption}

% en-tête pied-de-page
\usepackage{fancyhdr}

% colonnes
\usepackage{multicol}

% hyperliens
\usepackage{hyperref}
\usepackage{bookmark}
\hypersetup{
unicode=true,
pdftitle={Flore descriptive et illustrée de la France},
colorlinks=true
}

\ifpapierasix
\sloppy
\fi
\ifpapierasept
\sloppy
\fi

\renewcommand{\chapterheadstartvskip}{\vspace *{-\baselineskip}}
\renewcommand{\partformat}{\partname~\thepart}

\renewcommand{\sectionmark}[1]{\markright{\thesection ~ \ #1}}
\renewcommand{\chaptermark}[1]{\markboth{\chaptername\ \thechapter ~ \ #1}{}}

\renewcommand{\chapterformat}{}