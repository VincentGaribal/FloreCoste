% !TEX root = ../flore.tex

%  Macros de flore.tex

% charge les compteurs précédents
\InputIfFileExists{compteurs.aux}{}{}

% ouvre le fichier compteur.aux pour l'écriture
\newwrite\compteurfile
\immediate\openout\compteurfile = compteurs.aux

\ifpapieraquatre
\newcommand{\couleurBotaniste}{olive}
\newcommand{\couleurTermeGlossaire}{teal}
\newcommand{\couleurExempleGlossaire}{violet}
\fi
\ifpapieracinq
\newcommand{\couleurBotaniste}{olive}
\newcommand{\couleurTermeGlossaire}{teal}
\newcommand{\couleurExempleGlossaire}{violet}
\fi
\ifpapierasix
\newcommand{\couleurBotaniste}{black}
\newcommand{\couleurTermeGlossaire}{black}
\newcommand{\couleurExempleGlossaire}{black}
\fi
\ifpapierasept
\newcommand{\couleurBotaniste}{olive}
\newcommand{\couleurTermeGlossaire}{teal}
\newcommand{\couleurExempleGlossaire}{violet}
\fi

% Nom des taxons
\newcommand{\nomfamille}{}
\newcommand{\nomgenre}{}
\newcommand{\nomespece}{}

% Compteurs totaux
\newcounter{NombreTotalFamilles}
\newcounter{NombreTotalGenres}
\newcounter{NombreTotalEspeces}

% Compteurs hiérarchisés
\newcounter{NombreGenres}[NombreTotalFamilles]
\newcounter{NombreEspecesFamille}[NombreTotalFamilles]
\newcounter{NombreEspecesGenre}[NombreGenres]

% Singulier ou pluriel ?
\newcommand{\GenreSP}[1]{%
\ifthenelse%
{\cnttest{\csname the#1\endcsname}>{1}}%
{\numberstring{#1} genres}%
{\numberstring{#1} genre}%
}
\newcommand{\EspeceSP}[1]{%
\ifthenelse%
{\cnttest{\csname the#1\endcsname}>{1}}%
{\numberstring{#1} espèces}%
{\numberstring{#1}[f] espèce}%
}

%  Famille
% #1 : famille
\newcommand{\famille}[1]{%
\ifthenelse{\equal{\nomfamille}{}}{}{%
\immediate\write\compteurfile{\string\newcounter{NombreGenres\nomfamille Sauve}}
\immediate\write\compteurfile{\string\setcounter{NombreGenres\nomfamille Sauve}{\number\value{NombreGenres}}}
\immediate\write\compteurfile{\string\newcounter{NombreEspeces\nomfamille Sauve}}
\immediate\write\compteurfile{\string\setcounter{NombreEspeces\nomfamille Sauve}{\number\value{NombreEspecesFamille}}}}%
% Renomme la famille courante
\renewcommand{\nomfamille}{#1}%
% Incrémente le compteur de famille (donc remet à 0 les compteurs liés)
\refstepcounter{NombreTotalFamilles}%
\addpart[#1]{Famille des #1\label{#1}}%
\markboth{\nomfamille}{\nomfamille}
\begin{center}
\textit{Cette famille comporte \GenreSP{NombreGenres\nomfamille Sauve} et \EspeceSP{NombreEspeces\nomfamille Sauve}.}
\end{center}
\index[scientifique]{#1@\textbf{#1}}%
\thispagestyle{plain}%
\refstepcounter{coupletdetermination}%
}

% Genre
% #1 : genre
% #2 : botaniste
% #3 : nom vernaculaire
\newcommand{\genre}[3]{%
\newpage
\refstepcounter{NombreTotalGenres}
\ifthenelse{\equal{\nomgenre}{}}{}{%
\immediate\write\compteurfile{\string\newcounter{NombreEspeces\nomgenre Sauve}}
\immediate\write\compteurfile{\string\setcounter{NombreEspeces\nomgenre Sauve}{\number\value{NombreEspecesGenre}}}}
\refstepcounter{NombreGenres}
\renewcommand{\nomgenre}{#1}%
\markright{\nomgenre}% page courante
\addchap[#1]{Genre #1\index[scientifique]{#1}\ \botaniste{#2} \label{#1}\normalfont\ --- \emph{#3}\index[vernaculaire]{#3}}
\begin{center}
\emph{Ce genre comporte \EspeceSP{NombreEspeces\nomgenre Sauve}.}
\end{center}
\refstepcounter{coupletdetermination}}

% Espèce
% #1 : genre
% #2 : espèce
% #3 : botaniste
\newcommand{\espece}[3]{%
\clearpage
\renewcommand{\nomespece}{#2}%
\refstepcounter{NombreTotalEspeces}%
\refstepcounter{NombreEspecesGenre}
\refstepcounter{NombreEspecesFamille}
\addsec[#1 #2]{\emph{#1 #2}~\botaniste{#3}\index[scientifique]{#1!#2@\emph{#2}}\label{#1#2}}\noindent\rule[2ex]{\linewidth}{.4pt}\normalsize\normalfont}

% nom vernaculaire
\newcommand{\vernaculaire}[3]{\luadirect{vernaculaire(\luastring{#1},\luastring{#2},\luastring{#3})}}
\newcommand{\vernaculairelien}[2]{\hyperref[#2]{\color{\couleurExempleGlossaire}\emph{#1}}}

\renewcommand{\floatpagefraction}{0.35}
\setcounter{totalnumber}{5}

% raccourcis pour les sous-section de chaque espece
\newcommand{\biologie}{\normalfont\subsection*{Biologie}}
\newcommand{\ecologie}{\subsection*{Écologie (optimum écologiques)}}
\newcommand{\biogeographie}{\subsection*{Biogéographie}}

% raccourcis pour les nom des paragraphes
\newcommand{\appveg}{%
\paragraph[Appareil végétatif]{Appareil végétatif \subref{\nomgenre\nomespece veg}.}
\luadirect{tex.print(Flore[\luastring{\nomfamille}][\luastring{\nomgenre}][\luastring{\nomespece}]["Biologie"]["Appareil végétatif"])}%
}
\newcommand{\inflo}{%
\paragraph[Inflorescence]{Inflorescence \subref{\nomgenre\nomespece inflo}.}
\luadirect{tex.print(Flore[\luastring{\nomfamille}][\luastring{\nomgenre}][\luastring{\nomespece}]["Biologie"]["Inflorescence"])}%
}
\newcommand{\fleur}{%
\paragraph[Fleur]{Fleur \subref{\nomgenre\nomespece fleur}.}
\luadirect{tex.print(Flore[\luastring{\nomfamille}][\luastring{\nomgenre}][\luastring{\nomespece}]["Biologie"]["Fleur"])}%
}
\newcommand{\fruit}{%
\paragraph[Fruit]{Fruit \subref{\nomgenre\nomespece fruit}.}
\luadirect{tex.print(Flore[\luastring{\nomfamille}][\luastring{\nomgenre}][\luastring{\nomespece}]["Biologie"]["Fruit"])}%
}
\newcommand{\graine}{\paragraph{Graine.}}
\newcommand{\calice}{\textbf{Calice.\ }}
\newcommand{\corolle}{\textbf{Corolle.\ }}
\newcommand{\androcee}{\textbf{Androcée.\ }}
\newcommand{\gynecee}{\textbf{Gynécée.\ }}


% global lrbox
\makeatletter
\def\globallrbox#1{%
  \edef\reserved@a{%
    \endgroup
    \global\setbox#1\hbox{%
      \begingroup\aftergroup}%
        \def\noexpand\@currenvir{\@currenvir}%
        \def\noexpand\@currenvline{\on@line}}%
  \reserved@a
    \@endpefalse
    \color@setgroup
      \ignorespaces}
\def\endgloballrbox{\unskip\color@endgroup}
\makeatother

% Calcule la largeur de la dernière ligne du paragraphe
\newlength{\LargeurDerniereLigne}
\newsavebox{\Paragraphe}
\newsavebox{\DerniereLigne}
\newsavebox{\BoiteGlobale}
\newcommand{\LongueurDerniereLigne}[1]{%
\begin{lrbox}{\Paragraphe}
\parbox{\linewidth}{%
\noindent#1\par%
\setbox\DerniereLigne=\lastbox%
\begin{globallrbox}{\BoiteGlobale}%
\unhcopy\DerniereLigne
\end{globallrbox}%
}%
\end{lrbox}
\settowidth{\LargeurDerniereLigne}{\usebox\BoiteGlobale}}

\newlength{\valeurdeux}
\newlength{\somme}

\newcommand{\LigneCritereCle}[2]{%
\LongueurDerniereLigne{#1}
\settowidth{\valeurdeux}{#2}%
\setlength{\somme}{\LargeurDerniereLigne+3em+\valeurdeux}
\ifthenelse%
{\lengthtest{\somme > \linewidth}}%
{\sloppy\noindent#1\dotfill\\.\dotfill\mbox{#2}}%
{\sloppy\noindent#1\dotfill\mbox{#2}}}

\newcounter{coupletdetermination}
\newcounter{coupletdeterminationdiv}[coupletdetermination]

\newenvironment{cle}[1]{%
\begin{list}{}{%
\setlength{\itemsep}{0pt}%
\setlength{\parsep}{0pt}%
\refstepcounter{coupletdeterminationdiv}%
\renewcommand\makelabel{\color{red}{\textbf{\arabic{coupletdeterminationdiv}.}}\hfill}%
\phantomsection\label{cle#1}%
\settowidth\labelwidth{mm}%
\setlength\leftmargin{\labelwidth+\labelsep}}%
}%
{\end{list}\medskip}

\newcommand{\nom}[2]{%
\ifthenelse{\isempty{#2}}%
  {\textbf{\ref{#1}}}%
  {\mbox{\textbf{\hyperref[#1]{#2}}}}%
}

\newcommand{\itemcle}[2]{%
\item \LigneCritereCle{#1}{#2}
}

\ifpapieraquatre
\setlength{\columnseprule}{.4pt}
\setlength{\columnsep}{1cm}
\fi
\ifpapieracinq
\setlength{\columnseprule}{.4pt}
\fi
\ifpapierasix
\setlength{\columnseprule}{.4pt}
\fi

\newcommand{\formuleflorale}[1]{%
\begin{center}
\fbox{#1}
\end{center}}


\setcounter{tocdepth}{1}

% entrée des noms des botanistes
% #1 : nom abrégé
\newcommand{\entreebotaniste}[1]{\item[#1]\luadirect{tex.print(Botaniste[\luastringN{#1}])}.}

% affichage du nom complet du botaniste dans une bulle
\newcommand{\botaniste}[1]{%
\textcolor{\couleurBotaniste}{\pdftooltip{#1}{\luadirect{nomcompletbotaniste(\luastringN{#1})}}}}

% affichage de la définition du terme dans une bulle
% #1 : terme dans le texte
% #2 : terme dans le glossaire
% On peut utiliser le terme dans le texte modifié (nombre, majuscule, etc.).
% On précise alors dans le deuxième argument le terme tel qu'il est indiqué dans le glossaire
\newcommand{\glossaire}[2]{%
\textcolor{\couleurTermeGlossaire}{\pdftooltip{#1}{\luadirect{definitionglossaire(\luastringN{#2})}}}}

% entrée définitions du glossaire
% #1 : terme
% #2 : définition
\newcommand{\entreeglossaire}[2]{%
\paragraph[#1]{#1 ;}#2%
\luadirect{glossaire("#1",\luastringN{#2})}}

% description espèce
\newenvironment{descript}%
{\ifpapieraquatre
\begin{multicols}{2}
\fi
\ifpapieracinq
\begin{multicols}{2}
\fi}
{\ifpapieraquatre
\end{multicols}
\fi
\ifpapieracinq
\end{multicols}
\fi}

\newcommand{\dummybox}{%
\framebox[.49\linewidth]{\rule{0pt}{{.49\linewidth}*{4/3}}}
}

\newcommand{\icono}[8]{%
\ifpapieraquatre
\begin{figure*}
\ifthenelse{\equal{#1}{}}
  {\subfloat[Manquant\label{\nomgenre\nomespece veg}]{\dummybox}}
  {\subfloat[#2\label{\nomgenre\nomespece veg}]{\includegraphics[width=.49\linewidth]{#1}}}
\hfill
\ifthenelse{\equal{#3}{}}
  {\subfloat[Manquant\label{\nomgenre\nomespece inflo}]{\dummybox}}
  {\subfloat[#4\label{\nomgenre\nomespece inflo}]{\includegraphics[width=.49\linewidth]{#3}}}  

\ifthenelse{\equal{#5}{}}
  {\subfloat[Manquant\label{\nomgenre\nomespece fleur}]{\dummybox}}
  {\subfloat[#6\label{\nomgenre\nomespece fleur}]{\includegraphics[width=.49\linewidth]{#5}}}
\hfill
\ifthenelse{\equal{#7}{}}
  {\subfloat[Manquant\label{\nomgenre\nomespece fruit}]{\dummybox}}
  {\subfloat[#8\label{\nomgenre\nomespece fruit}]{\includegraphics[width=.49\linewidth]{#7}}}
\begin{center}
\bfseries\nomgenre\ \nomespece
\end{center}
\end{figure*}
\fi
}

% écologie

\newcommand{\lumiere}{%
\dimen0=\columnwidth\relax
\luadirect{barre(
\luastring{\nomfamille},
\luastring{\nomgenre},
\luastring{\nomespece},
"Lumière")}\par}

\newcommand{\humiditeatmospherique}{%
\dimen0=\columnwidth\relax
\luadirect{barre(
\luastring{\nomfamille},
\luastring{\nomgenre},
\luastring{\nomespece},
"Humidité atmosphérique")}\par}

\newcommand{\temperature}{%
\dimen0=\columnwidth\relax
\luadirect{barre(
\luastring{\nomfamille},
\luastring{\nomgenre},
\luastring{\nomespece},
"Température")}\par}

\newcommand{\continentalite}{%
\dimen0=\columnwidth\relax
\luadirect{barre(
\luastring{\nomfamille},
\luastring{\nomgenre},
\luastring{\nomespece},
"Continentalité")}\par}

\newcommand{\pH}{%
\dimen0=\columnwidth\relax
\luadirect{barre(
\luastring{\nomfamille},
\luastring{\nomgenre},
\luastring{\nomespece},
"pH")}\par}

\newcommand{\humidite}{%
\dimen0=\columnwidth\relax
\luadirect{barre(
\luastring{\nomfamille},
\luastring{\nomgenre},
\luastring{\nomespece},
"Humidité")}\par}

\newcommand{\texture}{%
\dimen0=\columnwidth\relax
\luadirect{barre(
\luastring{\nomfamille},
\luastring{\nomgenre},
\luastring{\nomespece},
"Texture")}\par}

\newcommand{\nutriments}{%
\dimen0=\columnwidth\relax
\luadirect{barre(
\luastring{\nomfamille},
\luastring{\nomgenre},
\luastring{\nomespece},
"Nutriments")}\par}

\newcommand{\salinite}{%
\dimen0=\columnwidth\relax
\luadirect{barre(
\luastring{\nomfamille},
\luastring{\nomgenre},
\luastring{\nomespece},
"Salinité")}\par}

\newcommand{\matiereorganique}{%
\dimen0=\columnwidth\relax
\luadirect{barre(
\luastring{\nomfamille},
\luastring{\nomgenre},
\luastring{\nomespece},
"Matière organique")}\par}